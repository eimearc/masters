\documentclass[12pt]{report}

\usepackage{geometry}
\usepackage{tabu}
\usepackage{dirtytalk}
\usepackage{graphicx}
\usepackage{url}
\usepackage{float}
\usepackage{listings}
\usepackage{algpseudocode}
\usepackage{algorithm}
\usepackage{algorithmicx}
\usepackage{subcaption}
\usepackage{verbatim}
\usepackage{amsmath}
\usepackage{wrapfig}
\usepackage{color}

\usepackage{amsthm}
\usepackage{verbatimbox}
\usepackage{multicol}
\usepackage[titletoc,title]{appendix}
\usepackage{amsfonts}
\usepackage[font={it}]{caption}

\usepackage{lmodern}

% For citing and referencing.
\usepackage{natbib}
\newcommand{\citebu}[1]{\citeauthor{#1} (\citeyear{#1})}

\geometry{
  a4paper,
  total={170mm,257mm},
  left=30mm,
  right=30mm,
  top=20mm,
  bottom=30mm,
}

\pagenumbering{roman}

\theoremstyle{definition}
\newtheorem{definition}{Definition}[section]

\begin{document}

  \renewcommand{\familydefault}{\sfdefault}
  \fontfamily{lmss}\selectfont

  \begin{titlepage}
    \centering
    {\Huge Bournemouth University\par}
    \vspace{0.5cm}
    {\Large National Centre for Computer Animation\par}
    \vspace{0.5cm}
    {\Large MSc in Computer Animation and Visual Effects\par}
    \vspace{5cm}
    {\huge \bfseries evulkan\par}
    \vspace{0.5cm}
    {\Large \bfseries \textit{A Vulkan Library}\par}
    \vspace{2cm}
    {\Large Eimear Crotty\par}
    % \includegraphics[width=0.15\textwidth]{images/MIT.png}\par\vspace{1cm}
    \vfill
    {\Large August 2020}
  \end{titlepage}

  \chapter*{Abstract}
    Vulkan is a graphics API which aims to provide users with faster draw speeds.
    The user is expected to explicitly provide the details previously given by
    the driver, as in the case of OpenGL. The resulting extra code can be
    difficult to understand and write at first, leading to the need for a 
    wrapper library.

  \chapter*{Acknowledgements}

    \vspace{1cm}
    % TODO.

  \tableofcontents

  \chapter{Introduction}
    \pagenumbering{arabic}
    Vulkan is a cross-platform graphics and compute API, developed by the
    Khronos Group. It aims to provide higher-efficiency than other current
    cross-platform APIs, by using the full performance available in today's
    largely-multithreaded machines. Vulkan achieves this by allowing tasks to be
    generated and submitted to the GPU in parallel (multithreaded programming).
    In addition, the API itself is written at a lower-level than other graphics
    APIs, meaning that the developer is required to provide many of the details
    previously generated by the driver at run-time.\\

    This project aims to alleviate this cost by providing a wrapper library for
    Vulkan, which allows a developer to use some of the more common features of
    Vulkan with much less effort than writing an application from scratch. This
    library is written in C++, using modern C++ features, adheres to both the
    official C++ Core Guidelines and Google C++ Style Guide and is fully unit
    tested. The library is available for download from GitHub and can be built
    using CMake.\\

    The library is specifically written with beginners and casual users of
    Vulkan in mind. The examples included in the repository provide a
    demonstration of how to use the library for different purposes, including
    drawing a triangle, loading an OBJ with a texture and using multiple passes
    to render simple objects with deferred shading.\\

    Testing123. \citebu{attiya1995sharing}. Another. \citebu{beyer2016site}

  \chapter{Previous Work}

  \chapter{Technical Background}

    \section{Limitations of OpenGL}
    OpenGL, the current cross-platform industry standard, was first released in 1992.

  \chapter{Solution}

  \chapter{Conclusion}

  \bibliographystyle{harvardnat}
  \bibliography{thesis.bib}

  \chapter*{Appendices}
  
\end{document}