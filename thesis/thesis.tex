\documentclass[12pt]{report}

\usepackage{geometry}
\usepackage{tabu}
\usepackage{dirtytalk}
\usepackage{graphicx}
\usepackage{url}
\usepackage{float}
\usepackage{listings}
\usepackage{algpseudocode}
\usepackage{algorithm}
\usepackage{algorithmicx}
\usepackage{subcaption}
\usepackage{verbatim}
\usepackage{amsmath}
\usepackage{wrapfig}
\usepackage{color}

\usepackage{amsthm}
\usepackage{verbatimbox}
\usepackage{multicol}
\usepackage[titletoc,title]{appendix}
\usepackage{amsfonts}
\usepackage[font={it}]{caption}

\usepackage{lmodern}

% For citing and referencing.
\usepackage{natbib}
\newcommand{\citebu}[1]{\citeauthor{#1} (\citeyear{#1})}

\geometry{
  a4paper,
  total={170mm,257mm},
  left=30mm,
  right=30mm,
  top=20mm,
  bottom=30mm,
}

\pagenumbering{roman}

\theoremstyle{definition}
\newtheorem{definition}{Definition}[section]

\begin{document}

  \renewcommand{\familydefault}{\sfdefault}
  \fontfamily{lmss}\selectfont

  \begin{titlepage}
    \centering
    {\Huge Bournemouth University\par}
    \vspace{1cm}
    {\Large National Centre for Computer Animation\par}
    \vspace{1cm}
    {\Large MSc in Computer Animation and Visual Effects\par}
    \vspace{4cm}
    {\huge\bfseries A Vulkan Library\par}
    \vspace{1.5cm}
    {\Large Eimear Crotty\par}
    % \includegraphics[width=0.15\textwidth]{images/MIT.png}\par\vspace{1cm}
    \vfill
    {\Large August 2020}
  \end{titlepage}

  \chapter*{Abstract}
    Vulkan is a graphics API which aims to provide users with faster draw speeds.
    The user is expected to explicitly provide the details previously given by
    the driver, as in the case of OpenGL. The resulting extra code can be
    difficult to understand and write at first, leading to the need for a 
    wrapper library.

  \chapter*{Acknowledgements}

    \vspace{1cm}
    % TODO.

  \tableofcontents

  \chapter{Introduction}
    \pagenumbering{arabic}
    Vulkan is a cross-platform graphics and compute API, developed by the Khronos Group. It aims to provide
    higher-efficiency than other current cross-platform APIs, such as OpenGL, by using the full performance
    available in today's largely-multithreaded machines.

    Testing123. \citebu{attiya1995sharing}. Another. \citebu{beyer2016site}

  \chapter{Previous Work}

  \chapter{Technical Background}

    \section{Limitations of OpenGL}
    OpenGL, the current cross-platform industry standard, was first released in 1992.

  \chapter{Solution}

  \chapter{Conclusion}

  \bibliographystyle{harvardnat}
  \bibliography{thesis.bib}

  \chapter*{Appendices}
  
\end{document}